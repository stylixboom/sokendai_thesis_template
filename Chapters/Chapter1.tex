% Chapter Template

\chapter{KODOMO Latex usage} % Main chapter title

\label{Chapter1}

\lhead{Chapter 1. \emph{Basic Latex usage by Siriwat K.}}



%----------------------------------------------------------------------------------------
%	SECTION 1
%----------------------------------------------------------------------------------------
\section{Introduction}
\label{intro}
First of all, welcome to Latex KODOMO guide. I will teach you very basic command that you can mimic. BUT, first I recommend you to download latex editor from here...\\ \\
\textbf{Windows}:\\
Prerequisite: MikTex \href{http://miktex.org/download}{http://miktex.org/download}\\
Editor: Texstudio \href{http://texstudio.sourceforge.net/}{http://texstudio.sourceforge.net}\\
\textbf{Mac/Ubuntu}:\\
Prerequisite: TexLive\\
Editor: Texstudio\\

After you meet all requirements, now you have to modify these file first..\\
1. Thesis.cls (Thesis title, your name, supervisor name)\\
2. main.tex (Committee, quote, abstract, acknowledge,...)\\

\textbf{If you don't need some page, e.g. quote, just delete it!!}

%----------------------------------------------------------------------------------------
%	SECTION 2
%----------------------------------------------------------------------------------------
\section{Basic figure}
\label{basic_fig}
This is how to reference to your figure. For example... we are talking about the following Fig. \ref{fig:SampleFigure}.

\begin{figure}[htbp]
	\centering
		\includegraphics[width=0.5\textwidth]{Figures/SampleFigure.pdf}
	\caption[Figure title here (will appear in the list)]{Figure caption here (will appear under figure)}
	\label{fig:SampleFigure}
\end{figure}

However, you have to prepare you image in .jpg, .png, .bmp, ... and the best is vector type image .pdf which can be exported from Word, Excel, Powerpoint.\\
1. File\\
2. Export to PDF/XLS\\



%-----------------------------------
%	SUBSECTION 1
%-----------------------------------
\subsection{Basic reference}
\label{basic_ref}
In order to reference, you have to use this command \textbf{\textbackslash ref\{your label here\} }. For example, as we described about how to insert figure, now we are talking about referencing (see section \ref{basic_fig}).



%-----------------------------------
%	SUBSECTION 2
%-----------------------------------
\subsection{Basic citation}
\label{basic_cite}
We normally need to cite another paper. To do that, we need this command \textbf{\textbackslash cite\{your bib key label here\}}. What is "bib"? It is an abbreviation Bibliography used in Latex. However, you need to create your references into the file "Bibliography.bib" before hand.

For example, our previous works \cite{BibKeyLabel1,BibKeyLabel2}... The good thing of cite and ref with Latex is, all numbering is automatic, and you can also click!! \cite{BibKeyLabel3}.

\clearpage % Start a new page


%----------------------------------------------------------------------------------------
%	CLOSING
%----------------------------------------------------------------------------------------
\section{In Closing}
\label{closing}
You have reached the end of this mini-guide. This should be veeeeeryyy basic enough. Anyway, you can find more how to use Latex by Googling! e.g. how to do newline, new page, bold text, italic, etc.

Good luck and have lots of fun!

\begin{flushright}
KODOMO Latex Guide written by ---\\
\textbf{Siriwat Kasamwattanarote}\\ \href{http://www.satoh-lab.nii.ac.jp/~stylix}{www.satoh-lab.nii.ac.jp/\textasciitilde stylix}\\
This template URL:\\ \href{http://www.satoh-lab.nii.ac.jp/~stylix/dl/sokendai_thesis_template_v1.zip}{www.satoh-lab.nii.ac.jp/\textasciitilde stylix/dl/sokendai\_thesis\_template\_v1.zip}
\end{flushright}